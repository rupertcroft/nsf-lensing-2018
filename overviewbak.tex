\section{Overview and Objectives}
Weak gravitational lensing has emerged as one of the powerful probes
of the structure of the Universe and ways of distinguishing
cosmological models. The small distortions of background images as
they are lensed by foreground matter are sensitive to both the matter
contents and the geometry of the Universe (e.g., \cite{blandford92},
\cite{hoekstra2008}).  Galaxy images are the most commonly used
cosmological sources (see \cite{Kilbinger2015} and references
therein), but weak lensing of the cosmic microwave background (CMB) is
also widely studied (see e.g., the review by \cite{lewis2006}, and
results from, e.g., the Planck satellite, Ade {\it et al.} 2015).
 
These two very successful implementations of lensing are similar in that they both are sensitive to -- and therefore can be used to measure -- the projected gravitational potential as a function of position on the sky:% {\bf
                                                                                                                                                                                                                          % this
                                                                                                                                                                                                                          % needs
                                                                                                                                                                                                                          % to
                                                                                                                                                                                                                          % be
                                                                                                                                                                                                                          % fixed}:
\begin{equation}
\Phi(\vec\theta)=\frac{2}{c^2}\int^{D(z_{s})}_{0}\frac{dD}{D(z_s)}\,\frac{D(z_{s},z)}{D}\phi(\vec
x),\eql{phi}
\end{equation}
where $\phi$ is the 3D potential at the position $\vec x=\vec
x[\vec\theta,D(z)]\rightarrow [D(z)\vec\theta,D(z)]$ in the small
angle limit; $D(z)$ is the angular diameter distance to redshift $z$
and $D(z,z')$ is the angular diameter distance between redshifts $z$
and $z'$. However, they differ not only in the sources that are
studied (galaxies in one case and the CMB in the other) but also in
the fundamental observable that is used to extract information about
$\Phi$: in one case, the distorted shapes of the galaxies are used to
infer information about the intervening mass, while in the case of the
CMB, the inhomogeneous potential leads to {\it anisotropies} in the
two-point statistics of the otherwise Gaussian field.



\Sfig{table}{Schematic overview of this proposal: shapes of source
  galaxies (upper left) and quadratic estimators that exploit the
  impact of lensing on the CMB two-point functions (middle panel) have
  matured in recent years so that they are used to estimate cluster
  masses (left figure in each panel) and make maps of the large scale
  structure of the universe (right figure in each panel). This
  proposal aims to expand the tool of lensing on two-point functions
  to the cases where galaxies (left, middle) and the Lyman alpha
  forest (right, middle) are the sources. Finally, it aims to use some
  of the same techniques to detect time delays from photons in the CMB
  (bottom row). In principle, time delays can also impact the
  two-point functions of galaxies (left, bottom) and the Lyman alpha
  forest (right, bottom), but we think the current focus of the
  proposal on the three shaded boxes presents a broad range of
  opportunities for students: ranging from guaranteed detections that
  could become competitive to more speculative ideas that will likely
  yield detections, but more work is needed to demonstrate that they
  can provide information competitive with the more traditional
  lensing methods.}

For the first probe, the \lya\ forest, we note that as the angular 
positions of quasars are deflected by the 
gravitational lensing effect of foreground matter, the \lya\ forest 
seen in the spectra of these quasars is 
therefore also lensed.
In \cite{croft17} (hereafter C18)
 the PI proposed that the 
\atf\ of the \lya\ forest
can also be measured.
%(e.g., \cite{zahn2006}). 
%As with 21cm data, 
The forest has the advantage of spectral information,
potentially yielding many lensed ``slices'' at different redshifts.

Our second new probe, the \atf\ 
of the angular galaxy distribution,
relies on the same physics. The positions
of galaxies are deflected by the gravitational potential produced by
foreground matter. This manifests itself in local distortions of
clustering statistics of qualitatively 
the same type which are measured in CMB lensing.
In the past, the number density of galaxies in large-area surveys 
was not sufficient to overcome the shot noise inherent in deriving
the lensing potential from the discrete galaxy distribution. This is
no longer the case however (as we show below, in this proposal), and
this offers a route to galaxy-based lensing constraints without
galaxy shape measurement.

Finally, as indicated in Fig.~\rf{table}, we propose to use \atf\ to
open up a new window on time delays. Until now, these have been
detected only in the case of strong lensing. But using the new
technique, we show that upcoming CMB experiments can estimate the time
delay field on the largest of scales. The projected potential
responsible for time delays differs from that in \ec{phi} in that it
does not contain the second ratio of distances in the integrand and
therefore its measurement probes the potential along a given line of
sight with a different weighting factor in redshift. Besides this
feature, the possibility of measuring properties of the universe on
the largest scales possible opens up windows on understanding
anomalies that have been observed at the 2-3 sigma level on large
scales.

We will carry out an in depth study of 
the weak lensing of these three new probes, spanning theory, 
simulations and first detections. 
We will study them together over the proposed period in order to benefit from
their common aspects, common
simulations and
the differing experience and perspectives of the PI and Co-PI.
Mock catalogs and  real data will be used to understand what can be achieved and to gain physical understanding
from it. Our ultimate aim is to develop these lensing tracers as new tools
for cosmology, to motivate 
observing strategies, thereby impacting future surveys, such as CMB-S4, DESI, and LSST. 



Our specific objectives are as follows:

(a) {\bf \underline{to simulate new lensing tracers,}} the Ly$\alpha$ forest and galaxy clustering,
 including non-linear physics, baryonic effects and observational
systematic errors.

(a) {\bf \underline{to develop statistical techniques}} for the analysis of new lensing data, including 
cross-correlations and mass reconstruction, and use knowledge from our full simulations to address and mitigate systematics.


(c) {\bf \underline{to make observational measurements}} spanning first 
detections to precision measurements of $\sigma_{8}$ at the 3\% level 
from different datasets, the CLAMATO, eBOSS and DES surveys.

(d) {\bf \underline{to explore new cosmological constraints}} from these
measurements which have different strengths and potential biases  to other
lensing results.

