The theory of general relativity dictates that particles
traveling through gravitational potential wells experience
time delays [1]. If two photons are emitted at the same
time, then they will travel different distances depending
upon the potential Φ through which they travel. In a

cosmological context of an expanding universe, the frac-
tional difference in comoving distance D to a source at

redshift z is equal to
d(ˆn) = −2
Z z
0
d ln(1 + z
0
) K(z, z0
)Φ (D(z
0
)ˆn;t(z
0
)) (1)

where the kernel K(z, z0
) = (H(z
0
)D(z))−1 with H the
Hubble rate; and t(z) is the age of the universe . There is
also a geometric time delay that is typically of the same
size for a single lens, but when the path is through a
series of peaks and troughs, the mean time delay is much
smaller, of order θRMS, so we neglect it here.

Similarly [2], photons that comprise the cosmic mi-
crowave background (CMB) experience these same time

delays or advances depending on the integrated potential

through which they travel.
This directional-dependent
change in the distance to last scattering is independent
of its finite width and a phenomenon different than the

angular deflections [3, 4] that have been captured by re-
cent experiments [5–9].

Although deflections and delays are two different phe-
nomena, they share some similarities, especially in the
case of the CMB. Both are determined by the integrated
potential along the line of sight, although with slightly
different kernels, as depicted in Figure 1. It is clear that

they will be highly correlated, so as a first approxima-
tion, we might view the maps of the lensing potential

created for example in Aghanim et al. [9] as maps of dis-
tance to the last scattering surface. Another similarity,

one that has not yet been exploited, is that the formalism
first proposed in Hu [3] can be applied to the delays as
well, and this is what we will carry out in this proposal.


The projected potential
responsible for time delays differs from that in \ec{phi} in that it
does not contain the second ratio of distances in the integrand and
therefore its measurement probes the potential along a given line of
sight with a different weighting factor in redshift. Besides this
feature, the possibility of measuring properties of the universe on
the largest scales possible opens up windows on understanding
anomalies that have been observed at the 2-3 sigma level on large
scales.
