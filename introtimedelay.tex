\subsection{Time Delays}

The theory of general relativity dictates that particles
traveling through gravitational potential wells experience
time delays [1]. If two photons are emitted at the same
time, then they will travel different distances depending
upon the potential Φ through which they travel. In a




The projected potential
responsible for time delays differs from that in \ec{phi} in that it
does not contain the second ratio of distances in the integrand and
therefore its measurement probes the potential along a given line of
sight with a different weighting factor in redshift. Besides this
feature, the possibility of measuring properties of the universe on
the largest scales possible opens up windows on understanding
anomalies that have been observed at the 2-3 sigma level on large
scales.
