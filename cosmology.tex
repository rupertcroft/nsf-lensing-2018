A significant part of our effort will be devoted to identifying ways 
in which our new lensing tracers can be advantageous to cosmology.
As an example, we will compute neutrino mass constraints using the
measurements. We will investigate how the new cluster mass determinations
(and the checks they provide on systematic errors in other methods)
propagate into estimates of dark energy parameters. In general, new probes
of the dark matter distribution which have no dependence on galaxy shape 
measurements, will have many advantages. The \lya\ forest with its automatic
full redshift information could also be used
to carry out tomography of the foreground matter, or measure the amplitude
of clustering at different redshifts (see the green lines in Figure 
\ref{pkpred}), and this can also be used to constrain dark energy. We note 
that  we  have not considered self-lensing by the forest (\cite{loverde2010}),
and this may offer a path to other, differing constraints.
