\section{Results from prior NSF support}

PI Croft is the PI of NSF award AST-1412966 (\$593K, 07/14-06/18
including no-cost extension), titled ``Gravitational redshifts of
galaxy clusters and large-scale structure: New probes of modified
gravity and dark matter''.  The project seeks to transform the study
of gravitational redshifts and their associated redshift distortions
into a major branch of cosmology.  {\it Intellectual merit:} The
results of the project so far are (1) predictions
for modified gravity and dark matter models and the effects of other
asymmetric redshift distortions.  (2) Formulating optimal statistical
estimators to measure gravitational redshifts.  (3) Making the 
first detection of
pairwise gravitational redshifts from large-scale structure. (4)
Making competitive constraints on deviations from General Relativity.
 These have been published as seven articles so far
\citep{2015MNRAS.453.1754A,2017MNRAS.471.2345Z,2017MNRAS.471.2077A,2017MNRAS.470.2822A,2017arXiv170907854G,2017MNRAS.465.4853A,2016MNRAS.456.3743A}. {\it
  Broader Impacts:} Apart from a total of 7 outreach presentations to
Middle School students, the White House Frontiers Conference and the
Allegheny observatory carried out so far, the main broad impact of the
project is through the dissemination of a Cosmology video
game. Six undergraduate students have worked on this part of the
project so far.


Co-I Scott Dodelson was co-I of NSF PHY-1125897: ``Physics Frontier Center at KICP'', and Award of \$22M spanning 2011--2017.
{\it Intellectual merit}
While supported by the Physics Frontier Center proposal, Dodelson
wrote 72 papers. Three of them are particularly relevant to this
proposal. First, he and student Eric Baxter led the effort to detect
CMB lensing around galaxy clusters using data from the South Pole
Telescope~\cite{Baxter:2014frs}. This 3-sigma detection set the stage
for CMB cluster lensing emerging as one of the key elements in the
proposal for CMB-S4 and more recently to yet another detection~\cite{Baxter:2017ixz}, from Dark Energy
Survey (DES). 
% The first detection using a likelihood formalism and
%the second a quadratic estimator. These are two of the techniques
%that will prove useful in the current proposal's aim to measure
%lensing-induced anisotropy in the galaxy correlation function.
A second was the first 
{\it  de-lensing} study of CMB
polarization~\cite{Manzotti:2017net}.
% Lensing of the CMB is a 
%phenomenon closely related to lensing-induced anisotropy of the galaxy
%correlation function, so understanding the broader context will help
%this proposal. 
Finally, Dodelson recently led the analysis that combined probes in
the DES to produce the most constraining results
yet from cosmic structure~\cite{Abbott:2017wau}. This effort is one
of the first of a new breed of cosmological analyses, where multiple
probes are combined; we expect the same from this proposal.
{\it Broader impacts} Dodelson was involved in the KICP outreach
activities over the lifetime of the grant, which involved     
Cosmology Short Courses - pioneering professional development for museum 
and planetarium staff designed to help bring current research into the 
museum and classroom.
 Museum Partnerships - collaborations with museums to create 
exhibits, shows, and programming that share cosmological research with 
broad and diverse audiences.
 Space Explorers - multifaceted, multi-year commitments
to inner city, minority, precollege students that included
 laboratory experiences, 
and residential science institutes at Yerkes Observatory

