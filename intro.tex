\section{Anistropies of the two point function}

chematic overview of this proposal: shapes of source
  galaxies (upper left) and quadratic estimators that exploit the
  impact of lensing on the CMB two-point functions (middle panel) have
  matured in recent years so that they are used to estimate cluster
  masses (left figure in each panel) and make maps of the large scale
  structure of the universe (right figure in each panel). This
  proposal aims to expand the tool of lensing on two-point functions
  to the cases where galaxies (left, middle) and the Lyman alpha
  forest (right, middle) are the sources. Finally, it aims to use some
  of the same techniques to detect time delays from photons in the CMB
  (bottom row). In principle, time delays can also impact the
  two-point functions of galaxies (left, bottom) and the Lyman alpha
  forest (right, bottom), but we think the current focus of the
  proposal on the three shaded boxes presents a broad range of
  opportunities for students: ranging from guaranteed detections that
  could become competitive to more speculative ideas that will likely
  yield detections, but more work is needed to demonstrate that they
  can provide information competitive with the more traditional
  lensing methods.

 Since so much of this proposal will be based on
the anisotropy in the 2-point statistics and because this idea is
relatively new \citep{Hu:2001tn} compared with the easier to
understand shape measurements, it is worth spending a paragraph
outlining the basic idea. Since the universe is homogeneous and
anisotropic, the two-point function of any observable, e.g., the
temperature on the surface of last scattering: $C(\phi)=\langle T(\vec
\theta)T(\vec\theta+\vec{\phi})\rangle$ depends only on the magnitude
of $\vec\phi$. When expanded in terms of spherical harmonics with
coefficients $a_{lm}$, this property leads directly to the relation
\be \langle a_{lm} a^*_{l'm'}\rangle = \delta_{ll'}\delta_{mm'} C_l
.\ee When photons from one part of the sky travel through an
over-dense region and from another through an under-dense region, the
situation changes and now the 2-point function depends not only on
angular distance between two points but also on the position on the
sky. The product of $a_{lm}$ and $a_{l'm'}$ now will have non-zero
expectation even when $l\ne l'$, an expectation value that is
proportional to the field that breaks the isotropy, $\Phi$. By forming
quadratic estimators with $lm$ and $l'm'$ slightly different, we can
obtain an estimate for the potential~\citep{Hu:2001tn,okamoto}. For
simplicity, we will refer to this effect as {\it anisotropic two-point
  functions}, or simply \atf.
 
\atf\ of the CMB field has proved fruitful, but it is not the only
possibility, as any light emitted at cosmological distances is
deflected on its way to us and therefore the statistics of any set of
sources will be anisotropic. With the huge growth in surveys of the
Universe we now have the exciting possibility of expanding the way
gravitational lensing is done and treating new fields and new
observations as sources. In the context of this more general view of
weak lensing, we have chosen two of the most promising sources to
focus on, the Lyman-alpha forest, and angular galaxy
clustering. Fig.~\rf{table} gives a schematic view of the new vistas
that can open up when we move beyond the CMB as a source.

\Sfig{table}{Schematic overview of this proposal: shapes of source
  galaxies (upper left) and quadratic estimators that exploit the
  impact of lensing on the CMB two-point functions (middle panel) have
  matured in recent years so that they are used to estimate cluster
  masses (left figure in each panel) and make maps of the large scale
  structure of the universe (right figure in each panel). This
  proposal aims to expand the tool of lensing on two-point functions
  to the cases where galaxies (left, middle) and the Lyman alpha
  forest (right, middle) are the sources. Finally, it aims to use some
  of the same techniques to detect time delays from photons in the CMB
  (bottom row). In principle, time delays can also impact the
  two-point functions of galaxies (left, bottom) and the Lyman alpha
  forest (right, bottom), but we think the current focus of the
  proposal on the three shaded boxes presents a broad range of
  opportunities for students: ranging from guaranteed detections that
  could become competitive to more speculative ideas that will likely
  yield detections, but more work is needed to demonstrate that they
  can provide information competitive with the more traditional
  lensing methods.}

For the first probe, the \lya\ forest, we note that as the angular 
positions of quasars are deflected by the 
gravitational lensing effect of foreground matter, the \lya\ forest 
seen in the spectra of these quasars is 
therefore also lensed.
In \cite{croft17} (hereafter C18)
 the PI proposed that the 
\atf\ of the \lya\ forest
can also be measured.
%(e.g., \cite{zahn2006}). 
%As with 21cm data, 
The forest has the advantage of spectral information,
potentially yielding many lensed ``slices'' at different redshifts.
An idealized test was carried out in C18 using
using a mock  high resolution angular grid of quasars (of order arcminute separation) and a linear theory foreground density
field. Standard quadratic estimators (e.g., \cite{okamoto})  
were used to successfully  reconstruct images of the foreground mass 
distribution. In the work proposed here we will expand the realism
of such tests and make measurements on real data. Enough work has been done to date on this source that this is a relatively low risk
project. There is still the question of how powerful a tool \atf\ of the \lya\ forest statistics will become. Will they surpass the more traditional shape measurements for at least some range of redshift? What are the systematics that must be treated in order to extract cosmological information. We argue below that we are well-suited to address these questions and feel that they provide a broad range of opportunities for graduate students.

Our second new probe, the \atf\ 
of the angular galaxy distribution,
relies on the same physics. The positions
of galaxies are deflected by the gravitational potential produced by
foreground matter. This manifests itself in local distortions of
clustering statistics of qualitatively 
the same type which are measured in CMB lensing.
In the past, the number density of galaxies in large-area surveys 
was not sufficient to overcome the shot noise inherent in deriving
the lensing potential from the discrete galaxy distribution. This is
no longer the case however (as we show below, in this proposal), and
this offers a route to galaxy-based lensing constraints without
galaxy shape measurement.

Finally, as indicated in Fig.~\rf{table}, we propose to use \atf\ to open up a new window on time delays. Until now, these have been detected only in the case of strong lensing. But using the new technique, we show that upcoming CMB experiments can estimate the time delay field on the largest of scales. The projected potential responsible for time delays differs from that in \ec{phi} in that it does not contain the second ratio of distances in the integrand and therefore its measurement probes the potential along a given line of sight with a different weighting factor in redshift. Besides this feature, the possibility of measuring properties of the universe on the largest scales possible opens up windows on understanding anomalies that have been observed at the 2-3 sigma level on large scales. 

We will carry out an in depth study of 
the weak lensing of these three new probes, spanning theory, 
simulations and first detections. 
We will study them together over the proposed period in order to benefit from
their common aspects, common
simulations and
the differing experience and perspectives of the PI and Co-PI.
Mock catalogs and  real data will be used to understand what can be achieved and to gain physical understanding
from it. Our ultimate aim is to develop these lensing tracers as new tools
for cosmology, to motivate 
observing strategies, thereby impacting future surveys, such as CMB-S4, DESI, and LSST. 

One final introductory comment related to the pair of pictures in the upper left and middle panels of Fig.~\rf{table}. The impact of lensing is easiest to detect in {\it cross-correlation}. The left-most picture in each of those panels depicts a detection of lensing due to galaxy clusters, in one case by measuring the tangential shear of background galaxies and in the other by measuring the \atf\ of the CMB behind clusters. The noise in the auto-correlation is larger, so in each case, the first detection came in the cross-correlation regime (with clusters as the lenses). We expect something similar in our cases, although throughout this proposal we will weave back and forth between the signal due to clusters and that due to large scale structure in general. Ultimately, we expect both types of signals to be detectable in all of our three  probes.

Our specific objectives are as follows:

(a) {\bf \underline{to simulate new lensing tracers,}} the Ly$\alpha$ forest and galaxy clustering,
 including non-linear physics, baryonic effects and observational
systematic errors.

(a) {\bf \underline{to develop statistical techniques}} for the analysis of new lensing data, including 
cross-correlations and mass reconstruction, and use knowledge from our full simulations to address and mitigate systematics.


(c) {\bf \underline{to make observational measurements}} spanning first 
detections to precision measurements of $\sigma_{8}$ at the 3\% level 
from different datasets, the CLAMATO, eBOSS and DES surveys.

(d) {\bf \underline{to explore new cosmological constraints}} from these
measurements which have different strengths and potential biases  to other
lensing results.

