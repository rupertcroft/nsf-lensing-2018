\section{Anisotropic two point functions}

The successful implementations of lensing so far (galaxy shear
and CMB \atf\ ) are similar in that 
they both are sensitive to -- and therefore 
can be used to measure -- the projected gravitational potential as a function 
of position on the sky:
\begin{equation}
\Phi(\vec\theta)=\frac{2}{c^2}\int^{D(z_{s})}_{0}\frac{dD}{D(z_s)}\,\frac{D(z_{s},z)}{D}\phi(\vec
x),\eql{phi}
\end{equation}
where $\phi$ is the 3D potential at the position $\vec x=\vec
x[\vec\theta,D(z)]\rightarrow [D(z)\vec\theta,D(z)]$ in the small
angle limit; $D(z)$ is the angular diameter distance to redshift $z$
and $D(z,z')$ is the angular diameter distance between redshifts $z$
and $z'$.

 Since so much of this proposal will be based on
the anisotropy in the 2-point statistics and because this idea is
relatively new \citep{Hu:2001tn} compared with the easier to
understand shape measurements, it is worth spending a paragraph
explaining the basic idea. Since the universe is homogeneous and
anisotropic, the two-point function of any observable, e.g., the
temperature on the surface of last scattering: $C(\phi)=\langle T(\vec
\theta)T(\vec\theta+\vec{\phi})\rangle$ depends only on the magnitude
of $\vec\phi$. When expanded in terms of spherical harmonics with
coefficients $a_{lm}$, this property leads directly to the relation
\be \langle a_{lm} a^*_{l'm'}\rangle = \delta_{ll'}\delta_{mm'} C_l
.\ee When photons from one part of the sky travel through an
over-dense region and from another through an under-dense region, the
situation changes and now the 2-point function depends not only on
angular distance between two points but also on the position on the
sky. The product of $a_{lm}$ and $a_{l'm'}$ now will have non-zero
expectation even when $l\ne l'$, an expectation value that is
proportional to the field that breaks the isotropy, $\Phi$. By forming
quadratic estimators with $lm$ and $l'm'$ slightly different, we can
obtain an estimate for the potential~\citep{Hu:2001tn,okamoto}.  This effect 
is what we call anisotropic two-point
 functions, \atf.
 
\atf\ of the CMB field has proved fruitful, but it is not the only
possibility, as any light emitted at cosmological distances is
deflected on its way to us and therefore the statistics of any set of
sources will be anisotropic. With the huge growth in surveys of the
Universe we now have the exciting possibility of expanding the way
gravitational lensing is done and treating new fields and new
observations as sources. In the context of this more general view of
weak lensing, we have chosen three of the most promising  to
focus on, the Lyman-alpha forest, angular galaxy
clustering, and CMB time delay anisotropies. Fig.~\rf{table} gives a 
schematic view of the new vistas that open up as a result.

 In principle, time delays can also impact the
  two-point functions of galaxies (Fig. ~\rf{table}, left, bottom) and 
the Lyman alpha
  forest (Fig. ~\rf{table}, right, bottom), but we think 
the current focus of the
  proposal on the three shaded boxes in Fig. ~\rf{table}
 presents a broad range of
  opportunities for students: ranging from guaranteed detections that
  could become competitive to more speculative ideas that will likely
  yield detections. In these latter cases, more work is needed to
 demonstrate that they
  can provide information competitive with the more traditional
  lensing methods.

A final introductory comment related to the pair of pictures in the
upper left and middle panels of Fig.~\rf{table}: The impact of lensing
is easiest to detect in {\it cross-correlation}. The left-most picture
in each of those panels depicts a detection of lensing due to galaxy
clusters, in one case by measuring the tangential shear of background
galaxies and in the other by measuring the \atf\ of the CMB behind
clusters. The noise in the auto-correlation is larger, so in each
case, the first detection came in the cross-correlation regime (with
clusters as the lenses). We expect something similar in our cases,
although throughout this proposal we will weave back and forth between
the signal due to clusters and that due to large scale structure in
general. Ultimately, we expect both types of signals to be detectable
in all of our three probes.

