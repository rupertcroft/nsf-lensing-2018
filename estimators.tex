\section{Planned work: Estimators}

Reconstructing the lensing mass distribution from observations
(and making maps, or measuring
clustering statistics or object properties directly) requires  an
estimator. 
A quadratic lensing estimator (e.g., \cite{okamoto}) 
is sensitive to variations in the power spectrum in  different 
regions of the sky.  These spatial variations result in
correlations in Fourier modes (or $C_\ell$'s) that would not
exist otherwise. 
We propose to further develop estimators
of the type that have been applied to the CMB but specialized to our
new tracers (for example the discrete sightlines of the \lya\ forest).


\subsection{Estimators for \lya\ forest lensing}
In C18 it was found that the standard quadratic estimator works 
very well on gridded spectral \lya\ forest data. As real
galaxies and quasars do not exist on a grid, the challenge is
to develop an estimator that works on a random discrete set of sightlines.
Several different approaches can be taken, the simplest being an interpolation
of the data onto a finely space uniform grid. We have experimented with this
approach, finding that it could be useful in conjunction with the detailed
simulation tests proposed above to establish how interpolation errors 
propagate. However, the PI, together with collaborator Ben Metcalf has 
developed a quadratic estimator which uses non-gridded data 
(\cite{metcalf17}). Without a grid, FFT methods are not available to 
speed up the estimator, but inversion of large matrices is required. At 
present tests on small (million pixel) simulations have shown success and
during this proposal we will improve the code and scale up to the capabilities
needed to process full mock and observational datasets.


\subsection{Estimators for anisotropic galaxy clustering}

In the case of \atf\ of galaxies
 we will be concentrating first on detecting and
measuring the lensing signal of stacked galaxy clusters..
The simplest estimator to try first is a parametric model with,
e.g., the mass and concentration of the clusters set as free
parameters. The data points will be   $N_{\rm pix}^2/2$
estimates of $\hat w$, and these
can be used to constrain the parameters.
We will also explore maximum likelihood methods
that do not rely on the small angle approximation.


\subsection{Estimators for Time Delay}

We will develop the time delay estimator shown schematically in \ec{tdest}. A preliminary estimate is that CMB-Stage IV will be able to detect this effect. However, quantifying the contamination in the time delay estimator coming from the deflection perturbation (which is much larger) is an important piece of this program. Two other routes suggest themselves: the auto-spectrum of the time delay field is difficult but feasible in our initial estimates, but the cross-spectrum between the time delay field and the lensing deflection field is another target, one that may be reached first. Second, we propose to develop estimators for the cross-correlation between massive objects such as clusters with the time delay field. This too may provide an earlier target for detection. Once detected, the time delay signal -- which peaks on very large scales -- may well shed light some of the more persistent anomalies that have been seen on large scales in COBE, WMAP, and Planck.

\subsection{Mitigating systematics: the road to precision cosmology}

Our simulation tests with realistic sources and lens fields will
reveal the extent to which non linearity and non Gaussianity affect
the lens reconstructions. We will  develop methods to mitigate
these effects. For example, in CMB lensing \cite{lewis16} have shown
how the effects of non Gaussian deflections from post-Born corrections
and non-linear structure growth can be can be severely reduced by
Gaussian smoothing. Also for CMB lensing, \cite{bohm18} have shown
that results from raytraced simulations can be accounted for and
corrected using analytic results (\cite{bohm16}). Although in the CMB
these issues are at the $\sim0.3 \%$ level in the power spectrum, they
will be a more significant issue with our lower redshift lensing
tracers and should be addressed as we seek to develop them as new
tools for precision cosmology. We will use the CMB lensing work as a
guide to push the precision of our work to below the level of the
simulated systematics.

\subsection{Cross-correlation and validation}
As our lensing techniques are new, an important part of our studies will
be checking results against relevant prior data.
Measurement of cross-correlations with foreground galaxies or CMB
lensing maps are likely to consistute the first proof that our
techniques are valid.
 In the case of forest lensing,
foreground galaxy maps are available for large areas of the sky from the
SDSS/BOSS and eBOSS surveys. We can use the distribution of galaxies 
(LRGs and ELGs from eBOSS) with redshifts,
weighted by the lensing kernel to make a predicted lensing
convergence  map. We will
 cross-correlate this map with convergence estimated from the forest. 
We will test this cross-correlation with our mock catalogs, and use
them to quantify systematic errors. In the case of LIA galaxy
clusters selected from photometry itself play the largest role,
and again the mocks will be important to bracket uncertainties 
(such as the role of baryonic physics  \cite{zentner2013}).


