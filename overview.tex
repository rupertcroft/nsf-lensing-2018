\section{Overview and Objectives}
Weak gravitational lensing has emerged as one of the powerful probes
of cosmology. The small distortions of background images as
they are lensed by foreground matter are sensitive to both the
contents and the geometry of the Universe \citep[e.g.,][]{blandford92,
hoekstra2008}.  Galaxy images are the most commonly used
cosmological sources (see \citealt{Kilbinger2015}, and references
therein), but weak lensing of the cosmic microwave background (CMB) is
also widely studied (see e.g., the review by \citealt{lewis2006}, and
results from, e.g., the Planck satellite, Ade {\it et al.} 2015).
 These two implementations of lensing differ not only 
in the sources 
studied  but also in
the fundamental observable. In one case,
 the shapes of galaxies are used to
determine the intervening gravitational potential (see the top
left panel in Figure 1, which shows galaxy clusters and large-scale structure).
On the other hand, for the
CMB, the potential can be mapped (middle panel of Figure 1) using
{\it anisotropies} in the
two-point statistics of the otherwise Gaussian field. 


These {\it anisotropic two-point functions} (hereafter \atf) will 
affect any cosmologically distributed background
sources. The field of weak lensing can therefore be broadened by looking 
at new applications of them. We choose three to work  on in this proposal:

{\bf(1) The \lya\ forest}- as the angular 
positions of quasars are deflected, the forest 
in quasar spectra is also lensed.
In \cite{croft17} (hereafter C18)
 the PI proposed measurement of the 
\atf\ of the \lya\ forest.


 {\bf (2) The \atf\ 
of the angular galaxy distribution}
relies on the same physics. The positions
of galaxies are deflected by lensing
resulting in local distortions of
clustering statistics.

{\bf (3) A new window on time delays with \atf}.So far, time delays have been
detected only in the case of strong lensing. But using our new
technique,  upcoming CMB experiments can estimate the time
delay field.



\Sfig{table}{\footnotesize{New vistas: anisotropies of the 2-pt. function 
in  galaxy clustering, CMB time delay, and the \lya\ forest}
}
\vspace{-0.1cm}


We will carry out an in depth study of these three new weak lensing
 applications of \atf\, spanning theory and 
simulations, from first detections to precision cosmology. We will study 
them together
 in order to benefit from
their common aspects, and common
simulations. Our specific objectives are as follows:

(a) {\bf \underline{to simulate new lensing tracers,}} the forest, galaxy clustering and time delay anisotropy,
 including non-linear physics, baryonic effects and observational
systematic errors.

(b) {\bf \underline{to develop statistical techniques}} for the analysis of new lensing data, including 
cross-correlations and mass reconstruction, and use knowledge from our full simulations to address and mitigate systematics.

(c) {\bf \underline{to make observational measurements}} from first 
detections to competitive  determinations of $\sigma_{8}$ at the 3\% level 
from forest and galaxy datasets, the CLAMATO, eBOSS and DES surveys.

(d) {\bf \underline{to explore new cosmological constraints}} from these
measurements which have different strengths and potential biases  to other
lensing results.

