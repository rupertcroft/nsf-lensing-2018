\section{Overview and Objectives}
Weak gravitational lensing 
has emerged as one of the powerful probes of the
structure of the Universe and ways of distinguishing cosmological models. The 
small distortions of background images as they are lensed by foreground matter
are sensitive to both the matter contents and the geometry of the Universe
(e.g., \cite{blandford92}, \cite{hoekstra2008}).
Galaxy images are the most commonly used cosmological 
sources (see \cite{Kilbinger2015} and references therein),
but weak lensing of the cosmic microwave background (CMB) 
is also widely studied (see e.g., the review by \cite{lewis2006}, and results from, e.g., the Planck satellite, Ade {\it et
 al.} 2015). 
 
These two very successful implementations of lensing are similar in that they both are sensitive to -- and therefore can be used to measure -- the projected gravitational potential as a function of position on the sky {\bf this needs to be fixed}:
\begin{equation}
\Phi(\vec\theta)=\frac{2}{c^2}\int^{z_{s}}_{0}dz\frac{D(z)D(z_{s},z)}{D(z_{s})}\phi(\vec x),\eql{phi}
\end{equation}
where $\phi$ is the 3D potential at the position $\vec x=\vec x[\vec\theta,D(z)]\rightarrow [D(z)\vec\theta,D(z)]$ in the small angle limit; $D(z)$ is the angular size distance to redshift $z$ and $D(z,z')$ is the angular size distance between redshifts $z$ and $z'$. However, they
differ not only in the sources that are studied (galaxies in one case and the CMB in the other) but also in the fundamental observable that is used to extract information about $\Phi$: in one case, the distorted shapes of the galaxies are used to infer information about the intervening mass, while in the case of the CMB, the inhomogeneous potential leads to {\it anisotropies} in the two-point statistics of the otherwise Gaussian field. 

Since so much of this proposal will be based on the anisotropy in the 2-point statistics and because this idea is relatively new \citep{Hu:2001tn}) compared with the easier to understand shape measurements, it is worth spending a paragraph outlining the basic idea. Since the universe is homogeneous and anisotropic, the two-point function of any observable, e.g., the temperature on the surface of last scattering: $C(\delta\theta)=\langle T(\vec \theta)T(\vec\theta+\vec{\delta\theta})\rangle$ depends only on the magnitude of $\vec\delta\theta$. When expanded in terms of spherical harmonics with coefficients $a_{lm}$, the temperature field then obeys the relation
\be
\langle a_{lm} a^*_{l'm'}\rangle = \delta_{ll'}\delta_{mm'} C_l
.\ee
When photons from one part of the sky travel through an over-dense region and from another through an under-dense region, the situation changes and now the 2-point function depends not only on angular distance between two points but also on the position on the sky. The coefficients now have non-zero expectation even when $l\ne l'$, an expectation value that is proportional to the field that breaks the isotropy, $\Phi$. By forming quadratic estimators with $lm$ and $l'm'$ slightly different, we can obtain an estimate for the potential. 
 
Anisotropies in the statistics of the CMB field has proved fruitful, but it is not the only possibility, as any light emitted
at cosmological distances is deflected on its way to us and therefore the statistics of any set of sources will be anisotropic. With the huge
growth in surveys of the Universe we now have the exciting possibility
of expanding the way gravitational lensing is done and treating new
fields and new observations as sources. In the context of this more general
view of weak lensing, we have chosen two of the most
promising sources to focus on, the Lyman-alpha forest, and 
angular galaxy clustering. Fig.~\rf{rable} gives a schematic view of the new vistas that can open up when we move beyond the CMB as a source.

\Sfig{table}{Schematic overview of this proposal: shapes of source galaxies (upper left) and quadratic estimators that exploit the impact of lensing on the CMB two-point functions (middle panel) have matured in recent years so that they are used to estimate cluster masses (left figure in each panel) and make maps of the large scale structure of the universe (right figure in each panel). This proposal aims to expand the tool of lensing on two-point functions to the cases where galaxies (left, middle) and the Lyman alpha forest (right, middle) are the sources. Finally, it aims to use some of the same techniques to detect time delays from photons in the CMB (bottom row). In principle, time delays can also impact the two-point functions of galaxies (left, bottom) and the Lyman alpha forest (right, bottom), but we think the current focus of the proposal on the three shaded boxes presents a broad range of opportunities for students: ranging from guaranteed detections that could become competitive to more speculative ideas that will likely yield detections, but more work is needed to demonstrate that they can provide information competitive with the more traditional lensing methods.}

For the first probe, the \lya\ forest, we note that as the angular 
positions of quasars are deflected by the 
gravitational lensing effect of foreground matter, the \lya\ forest 
seen in the spectra of these quasars is 
therefore also lensed.
In \cite{croft17} (hereafter C18)
 the PI proposed that the 
anisotropies in the 2-point function of the \lya\ forest
can also be measured.
%(e.g., \cite{zahn2006}). 
%As with 21cm data, 
The forest has the advantage of spectral information,
potentially yielding many lensed ``slices'' at different redshifts.
An idealized test was carried out in C18 using
using a mock  high resolution angular grid of quasars (of order arcminute separation) and a linear theory foreground density
field. Standard quadratic estimators (e.g., \cite{okamoto})  
were used to successfully  reconstruct images of the foreground mass 
distribution. In the work proposed here we will expand the realism
of such tests and make measurements on real data. Enough work has been done to date on this source that this is a relatively low risk
project. There is still the question of how powerful a tool anisotropies in the \lya\ forest statistics will become. Will they surpass the more traditional shape measurements for at least some range of redshift? What are the systematics that must be treated in order to extract cosmological information. We argue below that we are well-suited to address these questions and feel that they provide a broad range of opportunities for graduate students.

Our second new probe, the lensing-induced  anisotropy (LIA) 
of galaxy clustering,
relies on the same physics. The positions
of galaxies are deflected by the gravitational lensing effect of
foreground matter. This manifests itself in local distortions of
clustering statistics of qualitatively 
the same type which are measured in CMB lensing.
In the past, the number density of galaxies in large-area surveys 
was not sufficient to overcome the shot noise inherent in deriving
the lensing potential from the discrete galaxy distribution. This is
no longer the case however (as we show below, in this proposal), and
this offers a route to galaxy-based lensing constraints without
galaxy shape measurement.

Finally, as indicated in Fig.~\rf{table}, we propose to use lensing induced anisotropies to open up a new window on time delays. Until now, these have been detected only in the case of strong lensing. But using the new technique, we show that upcoming CMB experiments can estimate the time delay field on the largest of scales. The projected potential responsible for time delays differs from that in \ec{phi} in that it does not contain the ratio of distances and therefore its measurement probes the potential along a given line of sight with a different weighting factor in redshift. Besides this feature, the possibility of observing another feature of the universe on the largest scales possible opens up windows on understanding anomalies that have been observed at the 2-3 sigma level on large scales. 

Will carry out an in depth study of 
the weak lensing of these two new probes, spanning theory, 
simulations and first detections. 
We study both together over the proposed period in order to benefit from
their common aspects (e.g., the role of angular deflections), common
simulations and
the differing experience and perspectives of the PI and Co-PI.
Mock catalogs and  real 
data will be used
 to understand what can be achieved and to gain physical understanding
from it. Our ultimate aim is to develop these lensing tracers as new tools
for cosmology, and to motivate 
observing strategies.
Our specific objectives are as follows:

(a) {\bf \underline{to simulate new lensing tracers,}} the Ly$\alpha$ forest and galaxy clustering,
 including non-linear physics, baryonic effects and observational
systematic errors.

(a) {\bf \underline{to develop statistical techniques}} for the analysis of new lensing data, including 
cross-correlations and mass reconstruction, and use knowledge from our full simulations to address and mitigate systematics.


(c) {\bf \underline{to make observational measurements}} spanning first 
detections to precision measurements of $\sigma_{8}$ at the 3\% level 
from different datasets, the CLAMATO, eBOSS and DES surveys.

(d) {\bf \underline{to explore new cosmological constraints}} from these
measurements which have different strengths and potential biases  to other
lensing results.

