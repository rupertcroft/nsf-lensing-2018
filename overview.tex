\section{Overview and Objectives}
Weak gravitational lensing 
has emerged as one of the best ways to probe the
structure of the Universe and to test cosmological models. The 
small distortions
of background images as they are lensed by foreground matter
are sensitive to both the matter contents and the geometry of the Universe
(e.g., \cite{blandford92}, \cite{hoekstra2008}).
Galaxy images are the most commonly used cosmological 
sources (see \cite{Kilbinger2015} and references therein),
but weak lensing of the cosmic microwave background (CMB) 
is also widely studied (see e.g., the review by \cite{lewis2006}
, and results from the Planck satellite, Ade {\it et
 al.} 2015). These two types of dataset have met with great success
so far, but they are not the only possibilities, as any light emitted
at cosmological distances is lensed on its way to us. With the huge
growth in surveys of the Universe we now have the exciting possibility
of expanding the way gravitational lensing is done and treating new
fields and new observations as sources. In the context of this more general
view of weak lensing, we have chosen two of the most
promising sources to focus on, the Lyman-alpha forest, and 
angular galaxy clustering. 

For the first probe, the \lya\ forest, we note that as the angular 
positions of quasars are deflected by the 
gravitational lensing effect of foreground matter, the \lya\ forest 
seen in the spectra of these quasars is 
therefore also lensed.
In \cite{croft17} (hereafter C18)
 the PI proposed that the 
signature of weak gravitational lensing of the \lya\ forest
could be measured using similar techniques that have been applied to
the lensed Cosmic Microwave Background, and which have also been proposed for
application to  spectral data from 21cm radio telescopes 
(e.g., \cite{zahn2006}). 
As with 21cm data, the forest has the advantage of spectral information,
potentially yielding many lensed ``slices'' at different redshifts.
An idealized test was carried out in C18 using
using a mock  high resolution angular grid of quasars (
of order arcminute separation) and a linear theory foreground density
field. Standard quadratic estimators (e.g., \cite{okamoto})  
were used to successfully  reconstruct images of the foreground mass 
distribution. In the work proposed here we will expand the realism
of such tests and make measurements on real data.
 
Our second new probe, the lensing-induced  anisotropy (LIA) 
of galaxy clustering,
relies on the same physics. The positions
of galaxies are deflected by the gravitational lensing effect of
foreground matter. This manifests itself in local distortions of
clustering statistics of qualitatively 
the same type which are measured in CMB lensing.
In the past, the number density of galaxies in large-area surveys 
was not sufficient to overcome the shot noise inherent in deriving
the lensing potential from the discrete galaxy distribution. This is
no longer the case however (as we show below, in this proposal), and
this offers a route to galaxy-based lensing constraints without
galaxy shape measurement.

Will carry out an in depth study of 
the weak lensing of these two new probes, spanning theory, 
simulations and first detections. 
We study both together over the proposed period in order to benefit from
their common aspects (e.g., the role of angular deflections), common
simulations and
the differing experience and perspectives of the PI and Co-PI.
Mock catalogs and  real 
data will be used
 to understand what can be achieved and to gain physical understanding
from it. Our ultimate aim is to develop these lensing tracers as new tools
for cosmology, and to motivate 
observing strategies.
Our specific objectives are as follows:

(a) {\bf \underline{to simulate new lensing tracers,}} the Ly$\alpha$ forest and galaxy clustering,
 including non-linear physics, baryonic effects and observational
systematic errors.

(a) {\bf \underline{to develop statistical techniques}} for the analysis of new lensing data, including 
cross-correlations and mass reconstruction, and use knowledge from our full simulations to address and mitigate systematics.


(c) {\bf \underline{to make observational measurements}} spanning first 
detections to precision measurements of $\sigma_{8}$ at the 3\% level 
from different datasets, the CLAMATO, eBOSS and DES surveys.

(d) {\bf \underline{to explore new cosmological constraints}} from these
measurements which have different strengths and potential biases  to other
lensing results.

