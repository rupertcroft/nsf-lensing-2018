\section{Planned work: Simulations and mock catalogs}\label{sec:sim}

We will carry out a significant portion of our research using
 simulations and mock catalogs.
For the forest lensing we will build on our successful
linear theory tests (C18). For LIA
we propose to carry out detailed studies of the feasibility of
detection of the effect. First, we will refine the estimate 
in \S\ref{sec:lia} by accounting for: cosmic variance as an
additional source of noise (although this should not have too large
of an impact on the small scales of interest), more realistic cluster
profiles including mis-centering, the impact of magnification, the
validity of the Taylor expansion, optimal angular binning, and
photometric red shift errors and binning.



Unlike CMB lensing, the assumption of a initially Gaussian {\it source}
field is not a good one for our new weak lensing tracers. 
The foreground lens density field is also
at lower redshift that in the CMB case, meaning that the {\it lens}
density fluctuations will also be further into the non-linear regime. Mock
observations that include these non-linearities are therefore crucial
to test our methods. We will be working with a set of already run simulations
as well as carrying out some new ones to set up a comprehensive set
of realistic tests of lensing reconstruction and measurement. In this
section of the proposal we outline  how we plan to do this, starting
with the raw N-body and hydro simulations. In Table \ref{tab_runs} we list
some of the privately run simulations available to the project. We will
also make use of public simulation datasets such as \cite{giocoli2016}
 to supplement them.




\begin{table}[h]
\begin{center}
\footnotesize
\begin{tabular}{cccccc}

\hline\hline
  Run & N$_{\rm part}$   &   L$_{\rm box}$ &$\epsilon$&  $z_f$ & Physics \\
& & (Mpc/h) & (kpc/h)& & \\
  \hline
%  {\bf Completed} & & &  &  & \\
%  DM2448 & $2448^3$ & 400 & 6.5&  0.5 & Dark Matter\\
%  IGM 4096 & $2\times 4096^3$ & 400 & 3.25  & 2.0 & Dark Matter, Gas, Stars\\
%  IGM 2048 & $2\times 2048^3$ & 400 & 6.5 & 2.0 & Dark Matter, Gas, Stars\\
  {\it MassiveBlack-II} & $2\times 1792^3$ & 100 & 1.85 & 0 & DM, Gas, Stars, 
Blackholes\\
 {\it MassiveBlack-II  DM} & $2\times 1792^3$ & 100 & 1.85 & 0 & DM (complete)\\
\hline
  {\it Lyman-alpha} & $2\times 4096^3$ & 400 & 10.0&  1.5 & DM, Gas, 
Stars \\
%{\bf In Progress} & & & & & \\
{\it BlueTides DM} & $2\times 1758^3$ & 400 & 6.00 & 0 & DM  \\
{\it Zhu et al. DM} & $1024^3$ & 1000 & 10.00 & 0 & DM, 20 realizations  \\
\hline\\
\end{tabular}
\normalsize
\end{center}
\vspace{-1cm}
\caption{{\bf N-body Simulations available to this project.} 
  The columns denote the runs, the number of particles
  N$_{\rm part}$, the size of the simulation box  L$_{\rm box}$, the
  gravitational softening length $\epsilon$ and the final redshift 
to which it was run $z_f$. The
  last column denotes the physics involved. }
\label{tab_runs}
\end{table}
\vspace{-0.5cm}


\subsection{\lya\ source field}

Relevant \lya\ forest pixels for lensing will be mostly between 
 redshifts $z=2-3.5$, with the lower end of the range due to the atmospheric 
cutoff and the upper due to the decreasing availability of bright 
quasars and galaxies. Simulating the forest  involves resolving scales as
small as the pressure smoothing scale ($\sim 100 \kpc$, \cite{peeples10}) while
incorporating large scale modes on scales of tens of $\hmpc$.
We will model the forest using the following
three techniques which have different
strengths.

\subsubsection{Lognormal models}
In order to cover the largest volume, and make contiguous source
fields with diameters spanning large fractions of the sky (relevant for
surveys such as eBOSS and DESI), we will use lognormally transformed
linear density fields. This technique was introduced by \cite{bi1997}
and has been shown to be a good appproximation to full hydrodynamic
simulations for purposes such as measuring baryonic acoustic oscillations
\cite{legoff}. Large-scale velocity fields will be included via the
Zeldovich approximation, and thermal broadening added via a temperature
density relation \cite{keating17}.

\subsubsection{Hydrodynamic simulations}
Full hydrodynamic simulations will be useful to capture the effect of
small scale density fluctuations.
 Although volumes are relatively smaller
than what can be achieved with lognormal fields, the differences stemming
from
full simulation  in
lensing measurments from individual small areas will be measured without
cosmic variance by comparing the hydro simulations directly to lognormal
fields with the same random phases. In 
The PI has run  a $400\hmpc$ simulation 
(used in \cite{cisewski}) 
 with 
the Gadget-3 SPH code with full hydrodynamics, and this will be available to
this project. The volume spans 5.5 degrees at $z=2.5$ and this is enough
to simulate surveys several times larger than CLAMATO. Other
simulations are available with additional physics activated (see Table 
\ref{tab_runs})
including star formation and black holes.

\subsubsection{Combined models}
By mapping sections of hydrodynamic simulations onto a much larger
dark matter simulation (or even linear density fields) it is possible
to accurately reproduce many of the features of the \lya\ forest
in a larger hydrodynamic simulation. This
technique was used successfully by \cite{croft2004} and developed in
detail by \cite{peirani2014}. We have all the relevant 
simulation ingredients and will adopt this approch.

\subsection{\lya\ forest source Mocks}
\lya\ forest spectra will be made by integrating through the
neutral hydrogen distribution ( in SPH kernels in the case of the hydro
simulations)
along each sightline and then convolving with the line-of-sight velocities
and applying thermal broadening \citep{hernquist1996}.
 In order to make realistic
mock catalogs we will use the relevant observational parameters
of different datasets from Table\ref{obs}. Quasar and galaxy sightlines will
be picked randomly with the correct mean spacing. In our test so far
(C18) we have used a grid of background sources, which is obviously
unrealistic, and so an important part in our proposed work is to abandon
that simplification. Apart from survey geometry we will investigate
the role of uncertainties in  quasar and galaxy continuua. We will 
apply continua based on published principal components
of quasar spectra \citep{leedr9} and population synthesis models
to the mocks, before fitting them using low order polynomials. This will
enable us to estimate the covariance between pixels on large-scales which
are due to continuum fitting.


\subsection{Galaxy sources}
\label{galaxysourcesims}
In our study of LIA it is important to resolve large-scale structure
in the galaxy distribution but galaxy shapes themselves are not needed.
A number of suitable N-body simulations are therefore 
available to use, for example
those used in \cite{zhu2017} (see Table \ref{tab_runs}, and we will use a HOD 
approach (e.g,) to populate the dark matter distribution with galaxies.
Although angular clustering will be measured, we will make use of the 
three dimensional information in the simulations to assign photometric
redshifts and model systematic uncertainties related to their use in 
the presence of errors, catastrophic and otherwise \citep{hearin2010}.
Hydrodynamic simulations of small volumes are also available 
(Table \ref{tab_runs}) and
will be used as the source fields in lensing simulations of individual
galaxy clusters in order to investigate baryon physics effects.


\subsection{Simulated lenses}
In situations such as forest lensing with widely spaced sightlines,
the 
the lensing field will be detected only on large angular scales 
where the matter structure responsible for the lensing is linear to a
reasonably good approximation. Our lens simulations (e.g. C18)
have started from the simplest case, a Gaussian random field.   
We create a realization of the lensing potential by generating random 
Gaussian  distributed Fourier modes with the power spectrum expected 
in the same cosmology
 used to simulated source fields.   The deflection is found by taking the 
gradient of the potential using an FFT.  Points on the source planes 
are then displaced by the deflection to get the lensed image. 
In the case of the forest, the deflection is applied to individual pixels
(see Figure 1) in the forest source simulations
 and for LIA to galaxies described in Section \ref{galaxysourcesims}
We will start with a single lens plane approximation for the forest lensing
case based on the linear fields. For the LIA with a cluster lens, the 
simplest initial simulations will be of a single spherically  symmetric lens.

In order to capture the true non-linearity inherent in the lens field,
 we will then move to using ray-traced  Nbody simulations. We have chosen  
GLAMER   \cite{metcalf2014} to do this (collaborator Ben Metcalf is an 
author of the code). It incorporates adaptive mesh refinement for
efficient choice of ray shooting. The deflection and beam distortions 
(convergence and shear) are calculated by modified tree algorithm when haloes,
 point masses or particles are used and by fast Fourier transform when 
mass maps are used. The combination of these methods allow for a very 
large dynamical range, so that accurate  maps will be made
spanning several degrees and covering large-scale
structure in the lensing matter distribution. Multiple lens planes can be
handled by GLAMER \cite{petkova2014}, and the distribution of matter
will be taken from the simulations detailed in Table \ref{tab_runs}. A suite 
of publicly available raytraced simulations (the Multi Dark Lens Simulations,
\cite{giocoli2016}) has been carried out and will also be used for 
this project. These include over 150 realizations of $\sim 10$ square
degree fields carried out using fully sampled lightcone raytracing
with 24 lens planes each. The simulations therefore cover both the
redshift $z\sim1$ lenses relevant for forest lensing and the lower 
redshift needed for LIA in the same simulation. They can be used to
test both, and in principle look at the overlap in lensing reconstruction
(and lensing kernels) for the two methods. 


\subsection{Full mocks}

Given the simulated sources and lens described above, we will construct
full mock datasets where we mimic the geometry and noise properties
of particular datasets. These datasets are detailed below in Section 6.
We note that because some of them are extremely large in
angular extent (e.g. eBOSS and DES), we will make use of a combination of
small high fidelity mocks of sections (for example individual 
galaxy clusters) and others larger areas generated using the approximate 
techniques mentioned above.
We will also incorporate addition levels of
complexity and potential sources of systematic
error, treating the addition of each in turn. For both types of
lensing we consider, magnification will be important, as both source
galaxies and background quasars will be selected preferentially when
they are lensed. The effects of photometric redshift uncertainties
could also  be added 
at this stage.   
