\section{Project management and timeline}
The projects will be carried out by two graduate students (one
supported jointly by the grant an a TA position), 
the two PI/Co-PI faculty
members, outreach coordinator Turnshek, and undergraduate researchers.

\noindent{\bf Year 1}.  In the first year, the initial mock
datasets will be generated, both sources and lenses.
The PI and a graduate student
will coordinate and will be responsible for the 
forest lensing sources, the Co-I  and other graduate student will
work on the  galaxy source field.
dust. Both graduate students and undergraduates will combine these
with the lens fields. The PI and Co-I  and graduate students will
work on estimators, and start on first detections from CLAMATO an
a subsample of DES clusters. Turnshek will write the Minecraft lesson
plans, build the worlds with the 
UG students and test them on local schools 

\noindent{\bf Year 2}:
The graduate
student will lead a study of systematic selection and 
instrumental effects in the galaxy and forest
mocks. The PI will assemble a dataset drawn from
all SDSS eBOSS spectra taken up to that point, and start
scaling up the estimator code to handle millions of sightlines .
  The Co-I, graduate student and undergraduates will investigate
the predictions for clustering statistics from the mocks
and the potential for cosmological constraints. 
Turnshek will initiate the Universe Sandbox lesson plans. 

\noindent{\bf Year  3}: 
The PI and graduate student will make precise measurements of
the matter power spectrum from \lya\ lensing in the eBOSS
data.The Co-I will make  determinations of cluster masses
from the up-to-date DES data. Both PI and Co-I and  students
will evalulate cosmological constraints.
Turnshek will evaluate the success of
 the lesson plans and prepare them for wider distribution.

